\documentclass[12pt, titlepage]{article}

\usepackage{booktabs}
\usepackage{tabularx}
\usepackage{hyperref}
\hypersetup{
    colorlinks,
    citecolor=black,
    filecolor=black,
    linkcolor=red,
    urlcolor=blue
}
\usepackage[round]{natbib}

\title{SE 3XA3: Development Plan\\Title of Project}

\author{Team 12, DJS
	\\ Victor Velenchovsky - velech
	\\ Amandeep Panesar - panesas2
	\\ Taha Mian - miantm
}

\date{\today}



\begin{document}

\maketitle

\pagenumbering{roman}
\tableofcontents
\listoftables
\listoffigures

\begin{table}[bp]
\caption{\bf Revision History}
\begin{tabularx}{\textwidth}{p{3cm}p{2cm}X}
\toprule {\bf Date} & {\bf Version} & {\bf Notes}\\
\midrule
Date 1 & 1.0 & Notes\\
Date 2 & 1.1 & Notes\\
\bottomrule
\end{tabularx}
\end{table}

\newpage

\pagenumbering{arabic}

This document describes the requirements for ....  The template for the Software
Requirements Specification (SRS) is a subset of the Volere
template~\citep{RobertsonAndRobertson2012}.  If you make further modifications
to the template, you should explicity state what modifications were made.

\section{Project Drivers}

\subsection{The Purpose of the Project}
The purpose of this project is to make it easier for people that attend socail gatherings or events to select a songs and form their own playlist according to the mood or perference of the attendees.

Social gatherings are much more enjoyable when most of the attendee's enjoy
the music that is being played. This project was inspired by

\subsection{The Stakeholders}
Stakeholders will be people who attend these social events or gatherings and people who host them.

\textbf{Event Attendee's}
If the system is working properly and attendee's are voting for songs, then
the event will be more enjoyable for them.

\textbf{Event Organizer(s)}
Having users select music will put less stress on the event organizers, and
allow them to focus on other aspects of the event, or let them enjoy
themselves.

\subsubsection{The Client}
The client is the host of the social event or gathering, could also be a Dj whose not willing to put up with people fighting over what song to play next. Could also be the host whose trying to save money not hiring a Dj and just relying on this system that allows the attnedees to chose what song they want to play.

\subsubsection{The Customers}
Amandeeps Dad

\subsubsection{Other Stakeholders}
Other stakeholders could be music producers or licensors because music that is played should be legally attainted for the event, not pirated.

\subsection{Mandated Constraints}

\subsection{Naming Conventions and Terminology}

\subsection{Relevant Facts and Assumptions}

User characteristics should go under assumptions.

\section{Functional Requirements}

- Allow users to vote on which song to play next at a social gathering
- Users can select to vote for any of a pre-determined set of
  approximately 10 songs.
- The predetermined set of songs can be selected randomly from a larger pool
  of songs
- Queue up songs that have been voted on and automatically play them
- Provide a simple web-app for users to vote with. The web-app has a list that
  shows users what their options to pick from are

\subsection{The Scope of the Work and the Product}

\subsubsection{The Context of the Work}

\subsubsection{Work Partitioning}

\subsubsection{Individual Product Use Cases}

\subsection{Functional Requirements}

\section{Non-functional Requirements}

\subsection{Look and Feel Requirements}

- Visually pleasing web app interface

\subsection{Usability and Humanity Requirements}

- Straight-forward web app (new users should be able to adopt it easily)
- No sign up required
- Automatically connects to server via WiFi with little to no input from user

\subsection{Performance Requirements}

- Songs should play one after another with small or no delay in between
- Server should have high uptime

\subsection{Operational and Environmental Requirements}

\subsection{Maintainability and Support Requirements}

\subsection{Security Requirements}

- A user should only be allowed to vote once per \textit{'round'} of votes
- Restarting the app/WiFi/phone should not change the fact that a user can only
  vote once per \textit{'round'}

\subsection{Cultural Requirements}

\subsection{Legal Requirements}

\subsection{Health and Safety Requirements}

This section is not in the original Volere template, but health and safety are
issues that should be considered for every engineering project.

\section{Project Issues}

\subsection{Open Issues}

- How exactly do we make sure that a user can only vote once per song?

\subsection{Off-the-Shelf Solutions}

The project that we are modelling (PlayMyWay) already does most of what our
project will do.

As for other solutions, there are various libraries that we will use in our
development. These libraries will help with various functionality, and include
(but are not limited to):

- Express.JS (Server framework for Node.JS)
- Angular.JS (Front-end framework for the web app)
- nodeunit (Unit testing package)
- Mocha (general testing package)

\subsection{New Problems}

\subsection{Tasks}

\subsection{Migration to the New Product}

\subsection{Risks}

\subsection{Costs}

\subsection{User Documentation and Training}

\subsection{Waiting Room}

\subsection{Ideas for Solutions}

\bibliographystyle{plainnat}

\bibliography{SRS}

\newpage

\section{Appendix}

This section has been added to the Volere template.  This is where you can place
additional information.

\subsection{Symbolic Parameters}

The definition of the requirements will likely call for SYMBOLIC\_CONSTANTS.
Their values are defined in this section for easy maintenance.


\end{document}
