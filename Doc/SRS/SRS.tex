\documentclass[12pt, titlepage]{article}

\usepackage{booktabs}
\usepackage{tabularx}
\usepackage{hyperref}
\usepackage{float}
\hypersetup{
    colorlinks,
    citecolor=black,
    filecolor=black,
    linkcolor=red,
    urlcolor=blue
}
\usepackage[round]{natbib}
\usepackage{graphicx}
\title{SE 3XA3: Development Plan\\Title of Project}

\author{Team 12, DJS
	\\ Victor Velenchovsky - velech
	\\ Amandeep Panesar - panesas2
	\\ Taha Mian - miantm
}

\date{\today}

\begin{document}

\maketitle

\pagenumbering{roman}
\tableofcontents
\listoftables
\listoffigures

\begin{table}[bp]
\caption{\bf Revision History}
\begin{tabularx}{\textwidth}{p{3cm}p{2cm}X}
\toprule {\bf Date} & {\bf Version} & {\bf Notes}\\
\midrule
Wed. Oct. 5 & 0.1 & Basic Outline \\
Wed. Oct. 5 & 0.2 & Requirements added \\
Thurs. Oct. 6 & 0.3 & Section 1 added and formatting \\
Thurs. Oct. 6 & 0.4 & First draft \\
Thurs. Oct. 6 & 0.5 & Formatting and minor changes \\
Fri. Oct. 7 & 0.6 & First Revision complete \\
Sat. Oct. 8 & 0.7 & Section 4 Complete\\
Mon. Oct 11 & 1.0 & Revision 0\\
\bottomrule
\end{tabularx}
\end{table}

\newpage

\pagenumbering{arabic}

This document describes the requirements for ....  The template for the Software
Requirements Specification (SRS) is a subset of the Volere
template~\citep{RobertsonAndRobertson2012}.  If you make further modifications
to the template, you should explicity state what modifications were made.

\section{Project Drivers}

\subsection{The Purpose of the Project}
The purpose of this project is to make it easier for people that attend social
gatherings or events, to select songs and form their own playlist according to
the mood or preference of the attendees. The current implementation (PlayMyWay)
has an unflattering and difficult to use UI, as well as no easy way for the
average person to integrate the software into their party. We plan on making a
revised version that has an elegant web app interface, and an easy to install
server.

Social gatherings are much more enjoyable when most of the attendees enjoy
the music that is being played. This project was inspired by \\

Social gatherings are much more enjoyable when most of the attendees enjoy
the music that is being played. This project was inspired by

\subsection{The Stakeholders}

\subsubsection{The Client}

\textbf{Event Organizer}\\
The client is the host of the social event or gathering, who is trying to save
money by not hiring a DJ and simply relying on this system, which allows the
attendees to chose what songs they would like to hear. Having users select music
will put less stress on the event organizers, and allow them to focus on other
aspects of the event, or let them enjoy themselves. \\ \textbf{DJ} \\ The client
 can also be a DJ, hired by a party planner, who is not willing to put up with
  people fighting over what song to play next.

\subsubsection{The Customers}

\textbf{Event Attendee's}\\
If the system is working properly and attendee's are voting for songs, then
the event will be more enjoyable for them.
\subsubsection{Other Stakeholders}


No other stakeholders have been discovered.

\subsection{Mandated Constraints}

\begin{itemize}
\item The front-end of the product will take the form of a web-app that can
run on any javascript-enabled browser. This means the app should be able to
accommodate all common operating Systems (mobile and desktop) and all
common Javascript-enabled browsers. Internet Explorer may be exempt
\item The front-end web-app and server need to both be connected to the same
WiFi network.
\item The server needs to be stable with very low downtime, in order to prevent
scenarios where the music stops playing accidentally.
\end{itemize}

\subsection{Naming Conventions and Terminology}

\begin{table}[H]
\centering
\begin{tabular}{ c | c }
\textbf{Shorthand} & \textbf{Explanation}\\ \hline
JS & Javascript\\
HTML & Hypertext Markup Language\\
CSS & Cascading Style Sheets\\
Event & Any event that includes shared music listening. Party, Wedding, etc. \\
UI & Acronym for User Interface\\
BUC & Acronym for Business Use Case
\end{tabular}
\end{table}
\subsection{Relevant Facts and Assumptions}

\begin{itemize}
\item We assume that users of the app are impatient (they don't want to spend a
long time learning the software)
\end{itemize}

\section{Functional Requirements}

\subsection{The Scope of the Work and the Product}

\subsubsection{The Context of the Work}
\begin{figure}[!ht]
  \caption{A diagram of the context of work}
  \centering
    \includegraphics[width=0.5\textwidth]{contextWork.png}
\end{figure}
\subsubsection{Work Partitioning}

\begin{center}
\begin{table}[H]
\begin{tabularx}{\textwidth}{| X | X | X |}
\hline
\textbf{Event Name} &\textbf{Input and Output} &\textbf{Summary of BUC}  \\
\hline
Front Facing Website Interaction such as Voting & User Interaction (in) & Record the user's interaction (voting) on the website counting up\\
\hline
Front Facing Website Displays Votes  & Total Votes (out) & Transmit Total Votes To Front Facing Website \\
\hline
 Calculate Most Votes and Select Song & Total Votes (in)
Song Selected (out) & Get Total Votes and Show Song Selected On Front End\\
\hline
Reset Votes & Reset Votes (out) & Server should reset votes and counter \\
\hline
\end{tabularx}
\end{table}
\end{center}

\subsubsection{Individual Product Use Cases}

The project will be used primarily for playing music decided by a dynamic playlist. The playlist is generated by the server picking five unique random songs. The list of songs picked by the server is then shown to the user where any song can be picked. The voting system is based on most votes to one song, thus if a song is the most voted the product will play that song. After the highest voted song has been selected and has been playing for more than thirty seconds the votes will be reset and the next song will be picked using the same process.

\subsection{Functional Requirements}
\begin{itemize}
\item The web page created in HTML and Javascript will display graphics and other information to user like title.
\item The Javascript will fetch the five options for music playback and will display them on the user's browser.
\item The web page should also keep track of the number of votes for any song.
\item The html page should also include a graphic where the user can place their vote.
\item The html page should also only list valid options for music (should only list songs stored on the server).
\item Cookies will also be created to only allow one vote for the user.
\item After the song has been selected to play then the html page should update to only show songs that haven't been played and should reset votes.
\item The server should pick 5 random unique songs until no unique songs are left and repeat process.
\item The server should total votes and share the information with the user through the html page.
\item The server should sort the songs after voting and play the most voted song.
\item The cookie should also remember what song the user picked and then reset after voting has reset.
\end{itemize}

\section{Non-functional Requirements}

\subsection{Look and Feel Requirements}

\subsubsection{Appearance Requirements}

\begin{center}
\begin{table}[H]
\begin{tabularx}{\textwidth}{| c X |}
\hline
\multicolumn{2}{|c|}{\textbf{Non-Functional requirement}}\\
\hline
\textbf{Description : } & The product front end must be visually appealing and have no lag between interactions. \\
\hline
\textbf{Rationale : } & The user should not be frustrated because of poor design.\\
\hline
\textbf{Fit Criterion : } &  The user interface should be clean and simple. The buttons used must clearly be distinguishable and labels should be clear of spelling mistakes.\\
\hline
\end{tabularx}
\end{table}
\end{center}

\subsubsection{Style Requirements}

\begin{center}
\begin{table}[H]
\begin{tabularx}{\textwidth}{| c X |}
\hline
\multicolumn{2}{|c|}{\textbf{Non-Functional requirement}}\\
\hline
\textbf{Description : } & The product should have a modern UI interface with bright colors. \\
\hline
\textbf{Rationale : } & The colors will add to features and buttons being distinguishable. Furthermore, the product will create a brand by introducing custom color ( Facebook Blue).\\
\hline
\textbf{Fit Criterion : } &  The product should look like something from 2016 and not look like something from the 1990s html era.\\
\hline
\end{tabularx}
\end{table}
\end{center}

\subsection{Usability and Humanity Requirements}

\subsubsection{Ease of Use Requirements}

\begin{center}
\begin{table}[H]
\begin{tabularx}{\textwidth}{| c X |}
\hline
\multicolumn{2}{|c|}{\textbf{Non-Functional requirement}}\\
\hline
\textbf{Description : } & The product should be easy to use and not confuse users by having simple text describing how the product works.\\
\hline
\textbf{Rationale : } & Users might not know what or how to vote so a label or walkthrough might help them understand the uses and limitations of the product.\\
\hline
\textbf{Fit Criterion : } &  The product should have text explaining how voting works and how to vote for a certain song.\\
\hline
\end{tabularx}
\end{table}
\end{center}

\subsubsection{Personalization and Internationalization Requirements}

\begin{center}
\begin{table}[H]
\begin{tabularx}{\textwidth}{| c X |}
\hline
\multicolumn{2}{|c|}{\textbf{Non-Functional requirement}}\\
\hline
\textbf{Description : } & Any type of genre and song can be preloaded to the server.Furthermore the user can vote for any song they prefer.\\
\hline
\textbf{Rationale : } &  Users can ask event planners to preload the server with songs to allow users to personalize the song list.\\
\hline
\textbf{Fit Criterion : } &   Check the type of songs preloaded on the server.\\
\hline
\end{tabularx}
\end{table}
\end{center}

\subsubsection{Learning Requirements}

\begin{center}
\begin{table}[H]
\begin{tabularx}{\textwidth}{| c X |}
\hline
\multicolumn{2}{|c|}{\textbf{Non-Functional requirement}}\\
\hline
\textbf{Description : } & Should be easy to use.\\
\hline
\textbf{Rationale : } & All users should be able to easily vote for their favourite song.\\
\hline
\textbf{Fit Criterion : } & The product must show simple UI design to promote simplicity which allows for ease of use.\\
\hline
\end{tabularx}
\end{table}
\end{center}

\subsubsection{Understandability and Politeness Requirements}

\begin{center}
\begin{table}[H]
\begin{tabularx}{\textwidth}{| c X |}
\hline
\multicolumn{2}{|c|}{\textbf{Non-Functional requirement}}\\
\hline
\textbf{Description : } & The concept should be explained in the tutorial or instructions page clearly to allow understandability for the user.\\
\hline
\textbf{Rationale : } & All users should understand how the product works.\\
\hline
\textbf{Fit Criterion : } & The product must include an instructions, help, or walk-through page.\\
\hline
\end{tabularx}
\end{table}
\end{center}

\subsubsection{Accessibility Requirements}

\begin{center}
\begin{table}[H]
\begin{tabularx}{\textwidth}{| c X |}
\hline
\multicolumn{2}{|c|}{\textbf{Non-Functional requirement}}\\
\hline
\textbf{Description : } & The product should be easily accessed by users through the local wifi.\\
\hline
\textbf{Rationale : } & The users should be able to know where to go for the application to be able to use it.\\
\hline
\textbf{Fit Criterion : } & The URL that lets users votes must be short and easy to remember.\\
\hline
\end{tabularx}
\end{table}
\end{center}



\subsection{Performance Requirements}
\subsubsection{Speed and Latency Requirements}

\begin{center}
\begin{table}[H]
\begin{tabularx}{\textwidth}{| c X |}
\hline
\multicolumn{2}{|c|}{\textbf{Non-Functional requirement}}\\
\hline
\textbf{Description : } &The product should display and send votes the server quickly with no delay.\\
\hline
\textbf{Rationale : } & The users need to know if their song will be played next and the total amount of votes for the selected song.\\
\hline
\textbf{Fit Criterion : } & Product should show votes in real time with no delay or latency.\\
\hline
\end{tabularx}
\end{table}
\end{center}


\subsubsection{Safety-Critical Requirements}
Not Applicable
\subsubsection{Precision or Accuracy Requirements}

\begin{center}
\begin{table}[H]
\begin{tabularx}{\textwidth}{| c X |}
\hline
\multicolumn{2}{|c|}{\textbf{Non-Functional requirement}}\\
\hline
\textbf{Description : } &The song selected with the most amount of votes should be played.\\
\hline
\textbf{Rationale : } & The user expects that the most voted song should be played.\\
\hline
\textbf{Fit Criterion : } & The server should check to see if the song playing is the same as the most voted song.\\
\hline
\end{tabularx}
\end{table}
\end{center}


\subsubsection{Reliability and Availability Requirements}
\begin{center}
\begin{table}[H]
\begin{tabularx}{\textwidth}{| c X |}
\hline
\multicolumn{2}{|c|}{\textbf{Non-Functional requirement}}\\
\hline
\textbf{Description : } &  The server should always be playing music and should never have any down time \\
\hline
\textbf{Rationale : } & Users want to listen to music for the entire event or session. \\
\hline
\textbf{Fit Criterion : } & The product should record the uptime to see if the server ever goes down or resets. \\
\hline
%TABLE
\end{tabularx}
\end{table}
\end{center}
\subsubsection{Robustness or Fault-Tolerance Requirements}
\begin{center}
\begin{table}[H]
\begin{tabularx}{\textwidth}{| c X |}
\hline
\multicolumn{2}{|c|}{\textbf{Non-Functional requirement}}\\
\hline
\textbf{Description : } & The server should only play music that is available on the server.\\
\hline
\textbf{Rationale : } & The users should only be able to pick from music that has been preloaded onto the server. \\
\hline
\textbf{Fit Criterion : } & The product should only show list of music that is readily available on the server. \\
\hline
%TABLE
\end{tabularx}
\end{table}
\end{center}
\subsubsection{Capacity Requirements}
\begin{center}
\begin{table}[H]
\begin{tabularx}{\textwidth}{| c X |}
\hline
\multicolumn{2}{|c|}{\textbf{Non-Functional requirement}}\\
\hline
\textbf{Description : } & The server should be able to store 32GB of music.\\
\hline
\textbf{Rationale : } & The users want a wide variety of music. \\
\hline
\textbf{Fit Criterion : } & The product should have more then one song stored in the desginated folder. \\
\hline
%TABLE
\end{tabularx}
\end{table}
\end{center}
\subsubsection{Scalability or Extensibility Requirements}
\begin{center}
\begin{table}[H]
\begin{tabularx}{\textwidth}{| c X |}
\hline
\multicolumn{2}{|c|}{\textbf{Non-Functional requirement}}\\
\hline
\textbf{Description : } &The server should allow for atleast 300 users.\\
\hline
\textbf{Rationale : } &The events users will attend will consist of hundreds of guests and can reach into the thousands. \\
\hline
\textbf{Fit Criterion : } & The product should have more then one song stored in the desginated folder. \\
\hline
\end{tabularx}
\end{table}
\end{center}
\subsubsection{Longevity Requirements}
\begin{center}
\begin{table}[H]
\begin{tabularx}{\textwidth}{| c X |}
\hline
\multicolumn{2}{|c|}{\textbf{Non-Functional requirement}}\\
\hline
\textbf{Description : } &The web interface and server should always be up until the event coordinator manually turns off each service.\\
\hline
\textbf{Rationale : } & The user should be able to vote for songs until the event the user is attending is over. \\
\hline
\textbf{Fit Criterion : } & The product should not turn off until event coordinator specifies. \\
\hline
\end{tabularx}
\end{table}
\end{center}

\subsection{Operational and Environmental Requirements}
\subsubsection{Expected Physical Environment}
The physical environment does not effect non functional requirements
\subsubsection{Requirements for Interfacing with Adjacent Systems}
\begin{center}
\begin{table}[H]
\begin{tabularx}{\textwidth}{| c X |}
\hline
\multicolumn{2}{|c|}{\textbf{Non-Functional requirement}}\\
\hline
\textbf{Description : } &The web browser should fluently communicate with the server and record user interaction.\\
\hline
\textbf{Rationale : } & The web browser should let the server know what the user wants a certain song to be played. \\
\hline
\textbf{Fit Criterion : } & The server should show and log each interaction with the web browser. \\
\hline
\end{tabularx}
\end{table}
\end{center}
\subsubsection{Productization Requirements}
\begin{center}
\begin{table}[H]
\begin{tabularx}{\textwidth}{| c X |}
\hline
\multicolumn{2}{|c|}{\textbf{Non-Functional requirement}}\\
\hline
\textbf{Description : } &The product should have a runnable install script for easy distrubtion.\\
\hline
\textbf{Rationale : } & The admin should be able to install the product quickly \\
\hline
\textbf{Fit Criterion : } & The install script should install the product with no errors. \\
\hline
\end{tabularx}
\end{table}
\end{center}
\subsubsection{Release Requirements}
The product will only be released once unless an OS update corrupts the product


\subsection{Maintainability and Support Requirements}
\subsubsection{Maintenance Requirements}
\subsubsection{Supportability Requirements}
\subsubsection{Adaptability Requirements}
\begin{itemize}
\item Minimal required maintenance
\item Automatically pushed security updates
\end{itemize}

\subsection{Security Requirements}
\subsubsection{Access Requirements}

\begin{center}
\begin{table}[H]
\begin{tabularx}{\textwidth}{| c X |}
\hline
\multicolumn{2}{|c|}{\textbf{Only one vote}}\\
\hline
\textbf{Description : } & A user should only be allowed to vote once per 'round'
of votes.\\
\hline
\textbf{Rationale : } & A single user should only be able to vote once since
that's how democracy works\\
\hline
\textbf{Fit Criterion : } & Try and vote multiple times to check that it's not
possible\\
safeguards\\
\hline
\end{tabularx}
\end{table}
\end{center}

\begin{center}
\begin{table}[H]
\begin{tabularx}{\textwidth}{| c X |}
\hline
\multicolumn{2}{|c|}{\textbf{Restarting phone/WiFi cannot bypass voting}}\\
\hline
\textbf{Description : } & Restarting the phone or the WiFi should not allow
the user to vote multiple times\\
\hline
\textbf{Rationale : } & A single user should only be able to vote once since
that's how democracy works\\
\hline
\textbf{Fit Criterion : } & Try and vote multiple times by restarting phone,
logging out, restarting WiFi and see if any of these methods break the
safeguards\\
\hline
\end{tabularx}
\end{table}
\end{center}

\begin{center}
\begin{table}[H]
\begin{tabularx}{\textwidth}{| c X |}
\hline
\multicolumn{2}{|c|}{\textbf{User access}}\\
\hline
\textbf{Description : } & Only users attending the event should be able to vote\\
\hline
\textbf{Rationale : } & External actors should not influence the election\\
\hline
\textbf{Fit Criterion : } & Try to access the event without being connected to
the WiFi\\
\hline
\end{tabularx}
\end{table}
\end{center}

\subsubsection{Integrity Requirements}
\subsubsection{Privacy Requirements}

\begin{center}
\begin{table}[H]
\begin{tabularx}{\textwidth}{| c X |}
\hline
\multicolumn{2}{|c|}{\textbf{Private server data}}\\
\hline
\textbf{Description : } & No private server data should be visible to anyone
but the administrator\\
\hline
\textbf{Rationale : } & Server data should not be compromised\\
\hline
\textbf{Fit Criterion : } & Penetration testing\\
\hline
\end{tabularx}
\end{table}
\end{center}

\begin{center}
\begin{table}[H]
\begin{tabularx}{\textwidth}{| c X |}
\hline
\multicolumn{2}{|c|}{\textbf{Other user's data}}\\
\hline
\textbf{Description : } & No private data about other users should be visible
to anyone but the administrator\\
\hline
\textbf{Rationale : } & User's sensitive data should not be compromised\\
\hline
\textbf{Fit Criterion : } & Penetration testing\\
\hline
\end{tabularx}
\end{table}
\end{center}

\subsubsection{Audit Requirements}
\subsubsection{Immunity Requirements}

\subsection{Cultural Requirements}
There are no cultural requirements for this project.
\subsection{Legal Requirements}
\subsubsection{Compliance Requirements}

\begin{center}
\begin{table}[H]
\begin{tabularx}{\textwidth}{| c X |}
\hline
\multicolumn{2}{|c|}{\textbf{Non-Functional requirement}}\\
\hline
\textbf{Description : } &  Music that is played by someone should have legal rights to be played publicly \\
\hline
\textbf{Rationale : } & It is illegal to play music that you do not own the rights to play \\
\hline
\textbf{Fit Criterion : } & We use a website that offers Royalty free music for the testing \href{http://www.bensound.com/}{that can found here} \\
\hline
\end{tabularx}
\end{table}
\end{center}

\begin{center}
\begin{table} [H]
\begin{tabularx}{\textwidth}{| c X |}
\hline
\multicolumn{2}{|c|}{\textbf{Non-Functional requirement}}\\
\hline
\textbf{Description : } & Make sure that the original project we are trying to recreate allows us to look at the original project and take some things from their project. \\
\hline
\textbf{Rationale : } & We have to make sure we do not copy someone else's idea because it is illegal to copy work that you do not have the rights to. \\
\hline
\textbf{Fit Criterion : } & The open project we are recreating has an open MIT license \href{LICENSE.txt} {that can be found here.} We are allowed to use their project in any way we like.\\
\hline
\end{tabularx}
\end{table}
\end{center}

\subsubsection{Standards Requirements}
There are no standard requirements for this particular project.

\subsection{Health and Safety Requirements}


This section is not in the original Volere template, but health and safety are
issues that should be considered for every engineering project.

\section{Project Issues}

\subsection{Open Issues}

The most important issue right now is how exactly do we make sure that a user can only vote once per song, without requiring people to sign up for an account. We also want the user to be able to change their vote multiple times when selecting the next song and it is in queue.

\subsection{Off-the-Shelf Solutions}

The project that we are modeling (PlayMyWay) already does most of what our project will do.

As for other solutions, there are various libraries that we will use in our
development. These libraries will help with various functionality, and include
(but are not limited to):
\begin{itemize}
\item Express.JS (Server framework for Node.JS)
\item Angular.JS (Front-end framework for the webapp)
\item nodeunit (Unit testing package)
\item Mocha (general testing package)
\end{itemize}
\subsection{New Problems}

DJ.Js is based off the open source project called \href{https://github.com/malithsen/playmyway}{PlayMyWay that can be found here.} We are going to recreate the project, in javascript. The original project was written in Jade, which is a Object Oriented programming language based on Java. Jade is kind of outdated and not as universal as javascript, which is the golden standard for web page applications. We don't need any new installations just a device that can access the internet and has a internet browser that runs javascript.

\subsection{Tasks}

We are given an outline of the things we need for this project which includes a proof of concept, testing plan, a design, and a final presentation. The proof of concept will be early in the project but will allow us to make a very basic outline of the project like be able to get requests from a device to a server vice versa. The test plan will tell us how we are going to test our product so we know whether or not the project has successfully fulfilled our requirements The design of the project will be more specific to coding and will come later but can be broken down in these simple steps:
\begin{enumerate}
\item Build a node.js sever using Amandeeps Raspberry Pi for the
\item Build the UI using javascript
\item Testing (unit,general testing)
\end{enumerate}
The Final Presentation will be the last step in the project when it is complete and fully functional.
\subsection{Migration to the New Product}

There is no transition needed because we are recreating a web app, that already exists.

\subsection{Risks}

Many risks can occur while trying to implement our product which include:
\begin{itemize}
\item The product is a webapp and therefore relies on an internet connection
\item Bugs or catastrophic errors in the server could cause the music to start playing sporadically
\item Only people with a device that can connect to the internet can use the web app
\end{itemize}

\subsection{Costs}

There will be no monetary costs, the music we are using to test our webapp will be unlicensed and free to play, and because the project we are recreating is an open source project which we are free to use in any way. The only other cost is the amount of effort and time we put in the project which should not be more than 100 hours, but could take longer.

\subsection{User Documentation and Training}

We will integrate a small help button at the end of the webpage that will describe how the webpage works. After this short tutorial the user should know how the website works.

\subsection{Waiting Room}

Something that is part of our vision is having all genres of music, having a diverse list of music so users can select songs that cater to their tastes. We also want to add "moods" that correspond to certain event you are in, so a wedding would be more of a happy mood, where as a party will have a Festival will have a celebratory mood.

\subsection{Ideas for Solutions}
\begin{center}
\begin{table}[H]
\begin{tabularx}{\textwidth}{| c X |}
\hline
\multicolumn{2}{|c|}{\textbf{Non-Functional requirement}}\\
\hline
\textbf{Description : } & Make sure that the original project we are trying to recreate allows us to look at the original project and take some things from their project. \\
\hline
\textbf{Rationale : } & We have to make sure we do not copy someone else's idea because it is illegal to copy work that you do not have the rights to. \\
\hline
\textbf{Fit Criterion : } & The open project we are recreating has an open MIT license \href{LICENSE.txt} {that can be found here.} We are allowed to use their project in any way we like.\\
\hline
\end{tabularx}
\end{table}
\end{center}

\newpage

\bibliographystyle{plainnat}

\bibliography{SRS}

\newpage

\section{Appendix}

This section has been added to the Volere template.  This is where you can place
additional information.

\subsection{Symbolic Parameters}

The definition of the requirements will likely call for SYMBOLIC\_CONSTANTS.
Their values are defined in this section for easy maintenance.


\end{document}
