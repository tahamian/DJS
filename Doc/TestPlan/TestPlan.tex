\documentclass[12pt, titlepage]{article}
\usepackage{float}
\usepackage{booktabs}
\usepackage{tabularx}
\usepackage{hyperref}
\hypersetup{
    colorlinks,
    citecolor=black,
    filecolor=black,
    linkcolor=red,
    urlcolor=blue
}
\usepackage[round]{natbib}

\title{SE 3XA3: Test Plan\\DJS}

\author{Team 12 , DJS
		\\ Amandeep Panesar panesas2
		\\ Taha Mian miantm
		\\ Victor velech
}

\date{\today}



\begin{document}

\maketitle

\pagenumbering{roman}
\tableofcontents
\listoftables
\listoffigures

\begin{table}[bp]
\caption{\bf Revision History}
\begin{tabularx}{\textwidth}{p{3cm}p{2cm}X}
\toprule {\bf Date} & {\bf Version} & {\bf Notes}\\
\midrule
OCT 28 & 1.0 & Rev0\\

\bottomrule
\end{tabularx}
\end{table}

\newpage

\pagenumbering{arabic}


\section{General Information}

\subsection{Purpose}
The test plan document is a helpful tool for many large scale projects since it allows concise information about testing,verification, and validation geared towards the project. The following test cases were created for future references and allows the project to be implementated with testing and maintenance in mind. The test plan document will be updated before the project is fully implementated to allow for revision and any major changes involved.

\subsection{Scope}
The project, "DJS",is a democratic voting system which allows users to vote for music. Thus testing can cover many areas such as: client methods (ie: update song client, etc), server methods (ie: create Cookie), data structures, and sorting algorithms. 

\subsection{Acronyms, Abbreviations, and Symbols}

\subsection{Overview of Document}
\textcolor{red}{Making an issue to href something MAY be a better idea - CM} \\
This is the test plan document for the project DJS, which is a reconstruction of the application PlayMyWay \href{https://github.com/malithsen/playmyway}.
The test plan uses the functional and non-functional requirements to detect any errors in the project DJS.
The document goes over various techniques for testing such as Manual and automated testing, structural and
functional testing, static and dynamic testing, fault testing.

\section{Plan}
	
\subsection{Software Description}
The server running DJS is using nodejs with multiple libraries which includes : express, handlebars, express--handlebars, and socket.io. The implementation of DJS has been modularized into eight modules. The module Server.js is used for hosting the webpage, and uses the modules voter.js, player.js, metadata.js, home.js, library.js, args.js, and error-handler.js. server.js is the main module using every other module. Server.js uses Library.js to get an N amount of songs, the number N is set in the module server.js, the number has been preset to 5. All library.js does is get a list of songs that are to be displayed on the webpage to be voted for, and the songs are selected randomly. The same idea can be applied to metadata.js, except metadata.js gets the metadata of the songs selected by library.js, and saves the album in a folder called artwork. Voter.js also relies on server.js voter.js counts the amount of votes and server.js selects the song with the highest amount of votes. Home.js is used to display the webpage by server.js all the information like artwork, title of the song, and number of votes are all displayed using home.js. Sever.js uses player.js to just play audio to the speakers of the system. Error-handler.js is used by server.js for catching errors and determines what message to print to the terminal. Args.js is used by server.js when the user runs server.js with certain command line arguments.

\subsection{Test Team}
All project members will participate and be responsible for writing and executing tests.

\subsection{Automated Testing Approach}
\textcolor{red}{Speak to the why, where and how you will run automatic tests - CM} \\
The automated testing will be implemented by using javascript libaries and custom unit testing function created in javascript.
\subsection{Testing Tools}
\textcolor{red}{Should the reader know these? What are they for, purpose? I know Karma includes many features, which will you use? - CM} \\
The following testing libraries will be use: Selenium, Mocha, Karma, and Protractor. 

\subsection{Testing Schedule}
\textcolor{red}{Poor formatting, poor use of Gantt charts - CM} \\
Webpage should be operational by Oct 21/16
Server should be able to handle multiple users Oct 28/16
Voting system works by Oct 23/16
Songs in queue by Oct 25/16


\section{System Test Description}

\subsection{Tests for Functional Requirements}

\subsubsection{Client-side Graphical Interface}

\begin{center}
\begin{table}[H]
\begin{tabularx}{\textwidth}{| c X |}
\hline
\multicolumn{2}{|c|}{\textbf{Webpage Title and Buttons Loaded}}\\
\hline
\textbf{Type: } & Functional, Dynamic, Manual Testing\\

\textbf{Initial State: } & Web page is not loaded.\\

\textbf{Input: } & User's internet browser should navigate to the servers web address.\\

\textbf{Output: } & The server should serve the users request and load a webpage with a title and five buttons underneath.\\

\textbf{Test Procedure:  } &   The web page should be loaded and the title along with 5 buttons should be displayed to the user.\\
\hline
\end{tabularx}
\end{table}
\end{center}

\begin{center}
\begin{table}[H]
\begin{tabularx}{\textwidth}{| c X |}
\hline
\multicolumn{2}{|c|}{\textbf{Button Includes Song Title}}\\
\hline
\textbf{Type: } & Functional, Dynamic, Manual Testing\\

\textbf{Initial State: } & Web page is opened on users internet browser.\\

\textbf{Input: } & User's internet browser should navigate to the servers web address.\\

\textbf{Output: } & The webpage loaded should include five buttons with each button having text. The text inside each button should be of a different unique song title (each button has a song title).\\

\textbf{Test Procedure:  } & Load webpage on user internet browser and check if buttons have song titles (if test failed then output should be giberish on button).\\
\hline
\end{tabularx}
\end{table}
\end{center}



\begin{center}
\begin{table}[H]
\begin{tabularx}{\textwidth}{| c X |}
\hline
\multicolumn{2}{|c|}{\textbf{Vote Causes Button To Be Highlighted}}\\
\hline
\textbf{Type: } & Functional, Dynamic, Manual Testing\\

\textbf{Initial State: } & Web page is opened on users internet browser and buttons should be present with no prior votes.\\

\textbf{Input: } & User clicks on one button from the webpage.\\

\textbf{Output: } & The corresponding button selected will be highlighted in some form to indicate a vote has been cast and recorded .\\

\textbf{Test Procedure:  } & Load webpage on user internet browser and check if buttons have loaded. Once the buttons are present the tester selects one song and should result in the same button being highlighted.\\
\hline
\end{tabularx}
\end{table}
\end{center}

\begin{center}
\begin{table}[H]
\begin{tabularx}{\textwidth}{| c X |}
\hline
\multicolumn{2}{|c|}{\textbf{Graphic Object Shows Total Number Of Votes}}\\
\hline
\textbf{Type: } & Functional, Dynamic, Manual Testing\\

\textbf{Initial State: } & The web address is not loaded. The server has just started.\\

\textbf{Input: } & User navigates to web address.\\

\textbf{Output: } & The web page should load some graphical object which contains the number of votes for each corresponding button. The number of votes should be zero initially .\\

\textbf{Test Procedure:  } & The server should be freshly started. The tester should then navigate to the appropriate web url and load the web page. Once the web page has been loaded the tester can then observe the total number of votes.\\
\hline
\end{tabularx}
\end{table}
\end{center}

\subsubsection{Client-side Backend Interface}

\begin{center}
\begin{table}[H]
\begin{tabularx}{\textwidth}{| c X |}
\hline
\multicolumn{2}{|c|}{\textbf{Remeber Voted Song}}\\
\hline
\textbf{Type: } & Functional, Dynamic, Manual Testing\\

\textbf{Initial State: } & One song should have been voted and the internet browser closed.\\

\textbf{Input: } & The tester will place a vote on one random song and close the browser. After, the web page should be opened again by the tester and the page loaded.\\

\textbf{Output: } & The song title that was picked before closing the internet browser should be highlighted.\\

\textbf{Test Procedure: } & The tester will open a internet browser and load the webpage. After the webpage has been loaded the user will cast a vote. The internet browser opened previously will be closed. Then after the tester will reopen the internet browser and the song title that was selected previously should be highlighted.\\
\hline
\end{tabularx}
\end{table}
\end{center}


\begin{center}
\begin{table}[H]
\begin{tabularx}{\textwidth}{| c X |}
\hline
\multicolumn{2}{|c|}{\textbf{Song List Should Be Valid}}\\
\hline
\textbf{Type: } & Functional, Dynamic, Automated Testing\\

\textbf{Initial State: } & The web address is loaded. The server has just started.\\

\textbf{Input: } & The song titles that appears on website will be the input for the automated testing.Another input would be the music currently available on the server.\\

\textbf{Output: } & The unit testing function will return either with true or false. The result of true will indicate that the song list appeared on the web page matches the song titles available on the server.\\

\textbf{Test Procedure:  } & The automated test will record each song title generated and displayed on the client side. Furthermore, the songs available to the server will also be recorded. The result is calculated by matching all the songs recorded from the web page to the songs available to the server\\
\hline
\end{tabularx}
\end{table}
\end{center}



\subsubsection{Server-side Backend}

\begin{center}
\begin{table}[H]
\begin{tabularx}{\textwidth}{| c X |}
\hline
\multicolumn{2}{|c|}{\textbf{Create Cookie To Allow One Vote Per User}}\\
\hline
\textbf{Type: } & Structural, Dynamic, Automated Testing\\

\textbf{Initial State: } & The web address is not loaded. The server has just started.\\

\textbf{Input: } & A simulated user with random voting pattern that is active every couple of seconds.\\

\textbf{Output: } & The unit testing function will return true or false. The testing function will return true when the sum of total votes for each song  equals the number of users connected.Correspondingly the return value of false will suggest that one or more simulated users will have more then one vote.\\

\textbf{Test Procedure:  } & The automated test will create a certain number of random users. The server will create a cookie for each user that indicates a unique id to identify each user. The randomly generated users will all vote for one song that is picked randomly and then change all the votes to another random song ( ie. users 1..15 vote for song 1 then vote for song 2). The test function will then check the number of total votes for each song and sum them together which should equal the number of users generated. \\
\hline
\end{tabularx}
\end{table}
\end{center}

\begin{center}
\begin{table}[H]
\begin{tabularx}{\textwidth}{| c X |}
\hline
\multicolumn{2}{|c|}{\textbf{Reset Votes After Playing Song}}\\
\hline
\textbf{Type: } & Structural, Dynamic, Automated Testing\\

\textbf{Initial State: } & The number of total votes for a certain song is above zero.\\

\textbf{Input: } & The test function will need the total number of votes right after a certain song has been done playing.\\

\textbf{Output: } & The unit test function will return true or false. The test function will return true when the total number of votes after playing a song is zero.\\

\textbf{Test Procedure: } & The test procedure will start by having the webpage start with a song with the total number of votes above zero. The test function will then check after the song has played if the total number of votes is equal to zero. \\
\hline
\end{tabularx}
\end{table}
\end{center}

\begin{center}
\begin{table}[H]
\begin{tabularx}{\textwidth}{| c X |}
\hline
\multicolumn{2}{|c|}{\textbf{Check If Song List Is Unique}}\\
\hline
\textbf{Type: } & Structural, Dynamic, Automated Testing\\

\textbf{Initial State: } & The server started and web page loaded.\\

\textbf{Input: } & The test function will need the song list that is being sent to the client.\\

\textbf{Output: } & The unit test function will return true if the song list sent is unique and has no duplicates.\\

\textbf{Test Procedure: } &The test function will use the song list being sent to the client and store it into an array. As the song list for the client updates after a song has been played the new song list will be appended to the array. After the last song has played the test function will check the array to see if the server has sent any duplicate song titles and will result in a true or false value. \\
\hline
\end{tabularx}
\end{table}
\end{center}

\begin{center}
\begin{table}[H]
\begin{tabularx}{\textwidth}{| c X |}
\hline
\multicolumn{2}{|c|}{\textbf{Check If 5 Random Songs Picked}}\\
\hline
\textbf{Type: } & Structural, Dynamic, Automated Testing\\

\textbf{Initial State: } & The server started and web page loaded.\\

\textbf{Input: } & The test function will need to count the number of songs sent to the client after playing the current song.\\

\textbf{Output: } & The unit testing function will return true if the count is equal to five after playing the current song .\\

\textbf{Test Procedure: } &The test function will use a counter and check if the counter is equal to five after the current song is done being played. \\
\hline
\end{tabularx}
\end{table}
\end{center}

\begin{center}
\begin{table}[H]
\begin{tabularx}{\textwidth}{| c X |}
\hline
\multicolumn{2}{|c|}{\textbf{Play Most Voted Song}}\\
\hline
\textbf{Type: } & Structural, Dynamic, Automated Testing\\

\textbf{Initial State: } & The server started and web page loaded.\\

\textbf{Input: } & The test function will need to record the total number of votes and the corresponding song title picked.\\

\textbf{Output: } & The unit testing function will return true if the application plays the right song.\\

\textbf{Test Procedure: } &The testing function will use a counter and rank the songs by votes and check if the playing is song is equal to the song selected with the most votes. \\
\hline
\end{tabularx}
\end{table}
\end{center}


\subsection{Tests for Nonfunctional Requirements}
\textcolor{red}{ I don't like how these are structured. For example, NF Test 1: Initial State: Webpage has not been loaded. Input: User opens the webpage by doing X. Output: Webpage is open. How Test is Preformed: The user will rate on a scale... That is a better test because it is specific and repeatable. Also, make a survey and use symbolic parameters for everything (e.g. stress testing)  - CM} \\
\subsubsection{Look and Feel Requirements}
		
\paragraph{Appearance Tests}

\begin{enumerate}

\item{Non-Functional Requirement Test 1\\}

Type: Structural, Static, Manual \\
					
Initial State: Web page is loaded from a device that has Internet \\
					
Input/Condition: Users rate the web page on the ascetics of from a rating of one to 10. \\
					
Output/Result: The overall average of the results should be over 7.5. \\
					
How test will be performed: Users will take a short survey rating the Appearance, the results of multiple users will be
tabulated. The overall user average score will be taken, must have 20 plus users. \\

\end{enumerate}
\paragraph{Style Tests}

\begin{enumerate}			
\item{Non-Functional Requirement Test 2\\}

Type: Structural, Manual, Static etc.
					
Initial State: Web page is loaded from a device that has Internet \\
					
Input:  Users rate the web page on the ascetics of from a rating of one to 10. \\
					
Output:  The overall average of the results should be over 7.5. \\
					
How test will be performed: Users will take a short survey rating the Style, the results of multiple users will be
tabulated. The overall user average score will be taken, must have 20 plus users. \\

\end{enumerate}


\subsubsection{Usability and Humanity Requirements}

\paragraph{Ease of Use Requirements Test}

\begin{enumerate}

\item{ Non-Functional Requirement Test 3\\}

Type: Structural, Static, Manual
					
Initial State: Web page is loaded from a device that has Internet \\
					
Input/Condition: Users rate the web page on the Ease of Use from a rating of 1 to 10. \\
					
Output/Result: The overall average of the results should be over 7.5. \\
					
How test will be performed: Users will take a short survey rating the Ease of Use, the results of multiple users will be
tabulated. The overall user average score will be taken, must have 20 plus users. \\

\end{enumerate}

\paragraph{Understandability and Politeness Requirements Test}

\begin{enumerate}

\item{Non-Functional Requirement Test 4\\}

Type: Structural, Manual, Static
					
Initial State: Web page is loaded from a device that has Internet \\
					
Input/Condition: Users rate the tutorial on the web page that shows how it works on the  of from a rating of 1 to 10. \\
					
Output: The overall average of the results should be over 7.5. \\
					
How test will be performed: Users will take a short survey rating the tutorial, the results of multiple users will be
tabulated. The overall user average score will be taken, must have 20 plus user survey. \\
\end{enumerate}

\paragraph{Accessibility Requirements Test}

\begin{enumerate}

\item{Non-Functional Requirement Test 5\\}

Type: Structural, Manual, Static
					
Initial State: Web page is not loaded from a device that has Internet using local Wifi \\
					
Input/Condition: Users should be able to access the web page from local Wifi. \\
					
Output: The Web page is loaded on the device from local Wifi. \\
					
How test will be performed: The Web page will be loaded from 5 different local Wifi's every time the web page should load   \\

\end{enumerate}

\subsubsection{Performance Requirements}

\paragraph{Speed and Latency Requirements Test}

\begin{enumerate}

\item{ Non-Functional Requirement Test 6\\}

Type: Structural, Dynamic, Manual
					
Initial State: Web page is loaded from a device that has Internet \\
					
Input/Condition: There should be very little latency in loading the web page and making a vote \\
					
Output/Result: Web page should be loaded, Vote should be counted. \\
					
How test will be performed: It should take no longer than 3 seconds for the web page to load and to cast a vote, the time will be approximate so a stop watch will be enough to measure the latency times. \\
\end{enumerate}


\paragraph{Precision Test}

\begin{enumerate}

					
\item{Non-Functional Requirement Test 7\\}

Type: Structural, Manual, Dynamic
					
Initial State: Server is running \\
					
Input/Condition: The song with the most votes should be played next \\
					
Output: The song with the most votes is played next \\
					
How test will be performed: track the amount of votes and the songs for more than 50 song changes
and make sure that they are correct \\
\end{enumerate}


\paragraph{Reliability and Availability Requirement Test}

\begin{enumerate}

\item{Non-Functional Requirement Test 8\\}

Type: Structural, Manual, Static
					
Initial State: Server is running \\
					
Input/Condition: Server continues to run \\
					
Output: Server should constantly be playing music \\
					
How test will be performed: Let the server continuously run for a length period of time and check
put a mic next to it and detect whether or not the speaker playing music. Sound should be coming 
from the speaker all the time grace period of 50 seconds for in between songs \\
\end{enumerate}


\paragraph{Robustness Requirements Test}

\begin{enumerate}

\item{Non-Functional Requirement Test 9\\}

Type: Structural, Manual, Static
					
Initial State: Server is running \\
					
Input/Condition: Server plays next song(s) \\
					
Output: Next song(s) being played is from the sever only \\
					
How test will be performed: Users will take a short survey rating the tutorial, the results of multiple users will be
tabulated. The overall user average score will be taken, must have 20 plus user survey. \\
\end{enumerate}


\paragraph{Capacity Requirements Test}

\begin{enumerate}

\item{Non-Functional Requirement Test 10\\}

Type: Structural, Manual, Static
					
Initial State: Server is Running \\
					
Input/Condition: The system storage must be above 32 GB  \\
					
Output: The storage space is above 32 GB \\
					
How test will be performed: Inspect the system storage space (different for every OS) \\
\end{enumerate}

\paragraph{Scalability Requirements Test}

\begin{enumerate}

\item{Non-Functional Requirement Test 11\\}

Type: Structural, Manual, Static 
					
Initial State: Server is running \\
					
Input/Condition: Users using the server \\
					
Output: At least 300 users at a time \\
					
How test will be performed:Using Post man to generate multiple users 300 is max limit. \\

\end{enumerate}


\paragraph{Longevity Requirements Test}

\begin{enumerate}


\item{Non-Functional Requirement Test 12\\}

Type: Structural, Manual, Static
					
Initial State: Sever not running \\
					
Input/Condition: run the server \\
					
Output: Server should continue to run unless manual turn off \\
					
How test will be performed: leave server on for a lengthy period of time and then time the amount it
on for and come back and see if its still working \\

\end{enumerate}

\subsubsection{Operational and Environmental Requirements}

\paragraph{Requirements for Interfacing with Adjacent Systems Test}

\begin{enumerate}

\item{Non-Functional Requirement Test 13\\}

Type: Structural, Manual, Static
					
Initial State: Web page is not loaded from a device that has Internet \\
					
Input/Condition: Web page is loaded from a device that has Internet \\
					
Output: The web browser should fluently communicate with the server and record user interaction.  \\
					
How test will be performed: Users will take a short survey rating the tutorial, the results of multiple users will be
tabulated. The overall user average score will be taken, must have 20 plus user survey.    \\

\end{enumerate}

\paragraph{Productization Requirements Test}

\begin{enumerate}


\item{Non-Functional Requirement Test 14\\}

Type: Structural, Manual, Static
					
Initial State: Server not installed \\
					
Input/Condition: Runnable is installed \\
					
Output: Server is installed on system \\
					
How test will be performed: Runnable will be excecuted on multiple systems minimum 5 with different specs,
and each time should yeild installation sucessful \\
\end{enumerate}

\subsubsection{Maintainability and Support Requirements}

\paragraph{Access Requirements Test}

\begin{enumerate}


\item{Non-Functional Requirement Test 15\\}

Type: Structural, Manual, Dynamic
					
Initial State: Web page is loaded from a device that has Internet \\
					
Input/Condition: User clicks to vote \\
					
Output: Add one song voted for and total number of voters \\
					
How test will be performed: The total number of voters should not increase everytime a voter clicks
on another song. The amount of votes per song should change respective of the song last selected\\


\item{Non-Functional Requirement Test 16\\}

Type: Structural, Manual, Static
					
Initial State: Server is running, and user has been registered \\
					
Input/Condition: device is restarted or local wifi is restarted \\
					
Output: server should still hold user information \\
					
How test will be performed: A voter will load web page once vote and then restart there phone
and then load the web page again the total votes should not change, then do the same thing
but this time restart the wifi connection total votes should be constant \\
\item{Non-Functional Requirement Test 17\\}

Type: Structural, Manual, Static
					
Initial State: Web page is loaded from a device that has Internet \\
					
Input/Condition: Web page is loaded from a device not on local wifi \\
					
Output: Web page should not load web page \\
					
How test will be performed: will try and connect to the web page from another wifi connection \\
\end{enumerate}

\paragraph{Privacy Requirements Test}

\begin{enumerate}


\item{Non-Functional Requirement Test 18\\}

Type: Structural, Manual, Static
					
Initial State: Server is running \\
					
Input/Condition: User loads the web page  \\
					
Output: There should be no information about the server page to a user \\
					
How test will be performed: The sever will be running and a user will try and access the administrators webpage access should be denied \\

\item{Non-Functional Requirement Test 19\\}

Type: Structural, Manual, Static
					
Initial State: Web page is loaded from a device that has Internet \\
					
Input/Condition: User information is stored \\
					
Output: Other users should not have access to this information \\
					
How test will be performed: User will try and look at other user information on administrator page permission should be denied  \\
\end{enumerate}





\section{Tests for Proof of Concept}


	
\begin{center}
\begin{table}[H]
\begin{tabularx}{\textwidth}{| c X |}
\hline
\multicolumn{2}{|c|}{\textbf{Run Server}}\\
\hline
\textbf{Type: } & Structural, Dynamic, Manual Testing\\

\textbf{Initial State: } & Nothing Running .\\

\textbf{Input: } & Javascript Files.\\

\textbf{Output: } & Running Server.\\

\textbf{Test Procedure: } & The server should run when the command node file.js is ran. \\
\hline
\end{tabularx}
\end{table}
\end{center}
	
\begin{center}
\begin{table}[H]
\begin{tabularx}{\textwidth}{| c X |}
\hline
\multicolumn{2}{|c|}{\textbf{Play Music}}\\
\hline
\textbf{Type: } & Structural, Dynamic, Manual Testing\\

\textbf{Initial State: } & Server should be running and no music should be played .\\

\textbf{Input: } & Any song from the generated song list.\\

\textbf{Output: } & Music playing.\\

\textbf{Test Procedure: } & Run server and vote for any song. Then after votes have been counted the song with the most votes should be played. \\
\hline
\end{tabularx}
\end{table}
\end{center}
\begin{center}
\begin{table}[H]
\begin{tabularx}{\textwidth}{| c X |}
\hline
\multicolumn{2}{|c|}{\textbf{Load Buttons}}\\
\hline
\textbf{Type: } & Structural, Dynamic, Manual Testing\\

\textbf{Initial State: } & Nothing Running .\\

\textbf{Input: } & Javascript Files.\\

\textbf{Output: } & Running Server.\\

\textbf{Test Procedure: } & The server should run when the command node file.js is ran. \\
\hline
\end{tabularx}
\end{table}
\end{center}

\begin{center}
\begin{table}[H]
\begin{tabularx}{\textwidth}{| c X |}
\hline
\multicolumn{2}{|c|}{\textbf{Voting System}}\\
\hline
\textbf{Type: } & Structural, Dynamic, Manual Testing\\

\textbf{Initial State: } & The server should be running with the webpage loaded with no votes .\\

\textbf{Input: } & Vote .\\

\textbf{Output: } & Vote for song title .\\

\textbf{Test Procedure: } & The server should out put an array for now which shows the votes in a string array. \\
\hline
\end{tabularx}
\end{table}
\end{center}



	
\section{Comparison to Existing Implementation}	
\textcolor{red}{This is NOT just comparing the projects. It is detailing how to parallel test the two to be sure your project matches the other's (and subsequently your) requirements - CM} \\
The product DJS is a dynamic voting system which allows user to vote for songs. The previously implemented open source project called PlayMyWay lacked in well documented code. Comparing the two products is beneficial to DJS since it allows the product to evolve. The changes that might occur however are dependent on the project's progress and would require the scope to change. The modification to the scope would help develop and implement core features in DJS. The requirements that may add extra features will only be incorporated if time permits. The subsection of requirements such as look and feel, performance, security, and accuracy still need improvements. The section for look and feel for example is only partially implemented in the proof of concept. The requirement of displaying the total number of votes for each song will only be added if time permits. Another example, the performance requirement of loading buttons and quickly is still waiting to be implemented.  However, since most core requirements such as playing music and voting have been fulfilled most requirements suggested from before and others will be developed. Although most requirements are important  the list may also be narrowed down to help create more time for more ideal requirements such as total votes. Thus, after comparing DJS with the original implementation it demonstrates the similarities and the small differences still left to be negated.
				
\section{Unit Testing Plan}
		
\subsection{Unit testing of internal functions}
The implemented unit tests will help examine the product and will allow for clarity. The automated tests will be used to test a multitude of functional and nonfunctional requirements.  Majority of these tests will use the internal functions and variables implemented in the product. All tests will be correlated to core functions utilized in the product to ensure predictable outputs and behaviours for normal, abnormal, and negative scenarios.
		
\subsection{Unit testing of output files}		
The output for the DJS product will be the music playing and the client-side graphical interface. The unit testing for output will be done with the combination of both manual and automated testing. In addition, the testing of output will also include using an external library called selenium which helps javascript simulate clicks and other actions a user would commit. The tests will call functions that create the view and check if the view has appeared. An example would be is checking if the buttons have been loaded. The test for checking if buttons have appeared would involved the selenium library and would simulate buttons clicks. The unit tests will ensure that the proper methods are called and the output is the expected result. 
\bibliographystyle{plainnat}

\bibliography{SRS}

\newpage


\end{document}
