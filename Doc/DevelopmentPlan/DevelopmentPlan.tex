\documentclass{article}

\usepackage{booktabs}
\usepackage{tabularx}

\title{SE 3XA3: Development Plan\\Title of Project}

\author{Team 12, DJS
		\\ Student 1 name and macid
		\\ Amandeep Panesar - panesas2
		\\ Taha Mian - miantm
}

\date{\today}

\begin{document}

\begin{table}[hp]
\caption{Revision History} \label{TblRevisionHistory}
\begin{tabularx}{\textwidth}{llX}
\toprule
\textbf{Date} & \textbf{Developer(s)} & \textbf{Change}\\
\midrule
28/9/2016 & Amandeep & Added Git Workflow\\
29/9/2106 & Taha Mian & Added the Team communication/development plan(s)\\
... & ... & ...\\
\bottomrule
\end{tabularx}
\end{table}

\newpage

\maketitle

Good planning is an essential part of solving any problem, also an essential part of the engineering process. To work as a team efficiently and effectively we have to have develop plans that we should follow, and assign roles to each member.

\section{Team Meeting Plan}
The team meetings will be held every week on Thursday at Thode Library. In the meetings the group will discuss problems, milestones due, and plan a schedule for the following week.
\section{Team Communication Plan}
The team will mainly communicate which each other through a group chat in facebook. The group chat will be used to notify the group any time someone commmits on GitHub. We also have each other phone numbers if anything urgent comes up.
\section{Team Member Roles}
Team member roles will be governed by these rules:
\begin{enumerate}
\item There will be no team leader in our group, becasue we believe in equality.
\item We will have a scribe in the group and that will be Amandeep Panesar.
\item We will try and make sure the work the even as possible but we have some specializations
\begin{itemize}
\item Victor specializes in Javascript and Git. He will focus on both back end design and implementation.
\item Amandeep specializes in doucmentation and using a server. He will focus more on the design and implementing a server.
\item Taha specializes in LaTex and Javascript. He will focus on more of the front end design and implementation and the documentation
\end{itemize}
\end{enumerate}
\section{Git Workflow Plan}
The team will begin working on the open source project on Gitlab with the single repository. The contributors for the project will attempt to commit often and will only push refined code to the master branch.However, the development code will be implemented in private branches which will be merged when bugs, and problematic code is fixed. In addition, the code should have frequent commits with small changes to avoid merge conflicts and large rollbacks.
\section{Proof of Concept Demonstration Plan}
The proof of concept demonstation will display the server and the front end of the system. The front end of the project will only allow you to pick and play one song and will not include the voting aspect of the project. For the server aspect of the demonstration the song selected will play through the external speakers. The goal of this demonstration is to get the frontend client communicating with the server side and will allow us to verify that the project is feasible.
\section{Technology}
The software will be designed with primarily Node.JS and traditional web technologies.

Server: Express.JS
Back-end: Javascript (via Node.JS)
Web-app: HTML/CSS/JS as well as jQuery/AngularJS/Handlebars
UI: Bootstrap

\section{Coding Style}
We will be using the Javascript coding style found at:

http://javascript.crockford.com/code.html

\section{Project Schedule}
Provide a pointer to your Gantt Chart.

\section{Project Review}

\end{document}
