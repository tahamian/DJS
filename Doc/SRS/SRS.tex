\documentclass[12pt, titlepage]{article}

\usepackage{booktabs}
\usepackage{tabularx}
\usepackage{hyperref}
\hypersetup{
    colorlinks,
    citecolor=black,
    filecolor=black,
    linkcolor=red,
    urlcolor=blue
}
\usepackage[round]{natbib}

\title{SE 3XA3: Development Plan\\Title of Project}

\author{Team 12, DJS
	\\ Victor Velenchovsky - velech
	\\ Amandeep Panesar - panesas2
	\\ Taha Mian - miantm
}

\date{\today}

\begin{document}

\maketitle

\pagenumbering{roman}
\tableofcontents
\listoftables
\listoffigures

\begin{table}[bp]
\caption{\bf Revision History}
\begin{tabularx}{\textwidth}{p{3cm}p{2cm}X}
\toprule {\bf Date} & {\bf Version} & {\bf Notes}\\
\midrule
Wed. Oct. 5 & 0.1 & Basic Outline\\
Wed. Oct. 5 & 0.2 & Requirements added\\
\bottomrule
\end{tabularx}
\end{table}

\newpage

\pagenumbering{arabic}

This document describes the requirements for ....  The template for the Software
Requirements Specification (SRS) is a subset of the Volere
template~\citep{RobertsonAndRobertson2012}.  If you make further modifications
to the template, you should explicity state what modifications were made.

\section{Project Drivers}

\subsection{The Purpose of the Project}
The purpose of this project is to make it easier for people that attend social
gatherings or events to select a songs and form their own playlist according to
the mood or perference of the attendees. The current implementation (PlayMyWay)
has an unflattering and difficult to use UI, as well as no easy way for the
average person to integrate the software into their party. We plan on making a
revised version that has an elengant web app interface, and an easy to install
server.

Social gatherings are much more enjoyable when most of the attendee's enjoy
the music that is being played. This project was inspired by \\

Social gatherings are much more enjoyable when most of the attendee's enjoy
the music that is being played. This project was inspired by

\subsection{The Stakeholders}

\subsubsection{The Client}

\textbf{Event Organizer}\\
The client is the host of the social event or gathering, who is trying to save
money by not hiring a DJ and simply relying on this system, which allows the
attendees to chose what songs they want to listen to. Having users select music
will put less stress on the event organizers, and allow them to focus on other
aspects of the event, or let them enjoy themselves.

\textbf{DJ}\\
The client can also be a DJ, hired by a party planner, who is not willing to put
up with people fighting over what song to play next.

\subsubsection{The Customers}

\textbf{Event Attendee's}\\
If the system is working properly and attendee's are voting for songs, then
the event will be more enjoyable for them.

\subsubsection{Other Stakeholders}

No other stakeholders have been discovered.

\subsection{Mandated Constraints}

\begin{itemize}
\item The front-end of the product will take the form of a web-app that can
run on any javascript-enabled browser. This means the app should be able to
accomodate all common Operating Systems (phone and desktop) and all
common Javascript-enabled browsers. Internet Explorer may be exempt\\
\item The front-end web-app and server need to both be connected to the same
WiFi network.\\
\item The server needs to be stable with very low downtime, in order to prevent
scenarios where the music stops playing accidentaly.\\
\end{itemize}

\subsection{Naming Conventions and Terminology}

\begin{table}[bp]
\caption{\bf Naming Conventions and Terminology}
\begin{tabularx}{\textwidth}{p{3cm}p{2cm}X}
\toprule {\bf Shorthand} & {\bf Explanation}\\
\midrule
JS & Javascript\\
HTML & Hypertext Markup Language\\
CSS & Cascading Style Sheets\\
Event & Any event that includes shared music listening. Party, Wedding, etc.
\bottomrule
\end{tabularx}
\end{table}

\subsection{Relevant Facts and Assumptions}

\begin{itemize}
\item We assume that users of the app are mmpatient (they don't want to spend a
long time learning the software) \\
\end{itemize}

\section{Functional Requirements}

- Allow users to vote on which song to play next at a social gathering \\
- Users can select to vote for any of a pre-determined set of \\
  approximately 10 songs. \\
- The predetermined set of songs can be selected randomly from a larger pool \\
  of songs \\
- Queue up songs that have been voted on and automatically play them \\
- Provide a simple web-app for users to vote with. The web-app has a list that
  shows users what their options to pick from are \\

\subsection{The Scope of the Work and the Product}

\subsubsection{The Context of the Work}

\subsubsection{Work Partitioning}

\subsubsection{Individual Product Use Cases}

\subsection{Functional Requirements}

\section{Non-functional Requirements}

\subsection{Look and Feel Requirements}

- Visually pleasing web app interface

\subsection{Usability and Humanity Requirements}

\begin{itemize}
\item Straight-forward web app (new users should be able to adopt it easily)
\item No sign up required
\item Automatically connects to server via WiFi with little to no input from user
\end{itemize}
\subsection{Performance Requirements}

\begin{itemize}
\item Songs should play one after another with small or no delay in between
\item Server should have high uptime
\end{itemize}

\subsection{Usability and Humanity Requirements}

- Straight-forward web app (new users should be able to adopt it easily) \\
- No sign up required \\
- Automatically connects to server via WiFi with little to no input from user \\

\subsection{Performance Requirements}

- Songs should play one after another with small or no delay in between \\
- Server should have high uptime \\

\subsection{Operational and Environmental Requirements}

\subsection{Maintainability and Support Requirements}

\subsection{Security Requirements}

- A user should only be allowed to vote once per \textit{'round'} of votes \\
- Restarting the app/WiFi/phone should not change the fact that a user can only
  vote once per \textit{'round'} \\

\subsection{Cultural Requirements}

\subsection{Legal Requirements}
- Music that is played should have legal rights to be played publically \\
\subsection{Health and Safety Requirements}

This section is not in the original Volere template, but health and safety are
issues that should be considered for every engineering project.

\section{Project Issues}

\subsection{Open Issues}

- How exactly do we make sure that a user can only vote once per song, without
  requiring people to sign up for an account \\

\subsection{Off-the-Shelf Solutions}

The project that we are modelling (PlayMyWay) already does most of what our
project will do.

As for other solutions, there are various libraries that we will use in our
development. These libraries will help with various functionality, and include
(but are not limited to):
\begin{itemize}
\item Express.JS (Server framework for Node.JS)
\item Angular.JS (Front-end framework for the webapp)
\item nodeunit (Unit testing package)
\item Mocha (general testing package)
\end{itemize}

\subsection{New Problems}
DJ.Js is based off the open source project called \href{https://github.com/malithsen/playmyway}{playymway that can be found here.} We are going to recreate the project, in javascript. The original project was written in Jade, which is a Object Oriented programming language based on Java. Jade is kind of outdated and not as universal as javascript, which is the golden standard for web page applications.
\subsection{Tasks}
The project requires many steps to complete but here is a list:
\begin{enumerate}
\item 
\end{enumerate}
\subsection{Migration to the New Product}
There is no transition needed because this is a webapp, and is the first iteration of the product.
\subsection{Risks}
-The product is a webapp and therefore relies on an internet connection. \\
- Bugs or catastrophic errors in the server could cause the music to start
  playing sporadically \\
- Only people with a device that can connect to the internet can use the webapp\\
- 

\subsection{Costs}

\subsection{User Documentation and Training}

\subsection{Waiting Room}
Something that is part of our vision is having all genres of music having a diverse list of music so users can select songs that caiter to their tastes.
\subsection{Ideas for Solutions}

\bibliographystyle{plainnat}

\bibliography{SRS}

\newpage

\section{Appendix}

This section has been added to the Volere template.  This is where you can place
additional information.

\subsection{Symbolic Parameters}

The definition of the requirements will likely call for SYMBOLIC\_CONSTANTS.
Their values are defined in this section for easy maintenance.


\end{document}
