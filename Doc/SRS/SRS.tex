\documentclass[12pt, titlepage]{article}

\usepackage{booktabs}
\usepackage{tabularx}
\usepackage{hyperref}
\hypersetup{
    colorlinks,
    citecolor=black,
    filecolor=black,
    linkcolor=red,
    urlcolor=blue
}
\usepackage[round]{natbib}
\usepackage{graphicx}
\title{SE 3XA3: Development Plan\\Title of Project}

\author{Team 12, DJS
	\\ Victor Velenchovsky - velech
	\\ Amandeep Panesar - panesas2
	\\ Taha Mian - miantm
}

\date{\today}

\begin{document}

\maketitle

\pagenumbering{roman}
\tableofcontents
\listoftables
\listoffigures

\begin{table}[bp]
\caption{\bf Revision History}
\begin{tabularx}{\textwidth}{p{3cm}p{2cm}X}
\toprule {\bf Date} & {\bf Version} & {\bf Notes}\\
\midrule
Wed. Oct. 5 & 0.1 & Basic Outline \\
Wed. Oct. 5 & 0.2 & Requirements added \\
Thurs. Oct. 6 & 0.3 & Section 1 added and formatting \\
Thurs. Oct. 6 & 0.4 & First draft \\
Thurs. Oct. 6 & 0.5 & Formatting and minor changes \\
Fri. Oct. 7 & 0.6 & First Revision complete \\
Sat. Oct. 8 & 0.7 & Section 4 Complete\\
\bottomrule
\end{tabularx}
\end{table}

\newpage

\pagenumbering{arabic}

This document describes the requirements for ....  The template for the Software
Requirements Specification (SRS) is a subset of the Volere
template~\citep{RobertsonAndRobertson2012}.  If you make further modifications
to the template, you should explicity state what modifications were made.

\section{Project Drivers}

\subsection{The Purpose of the Project}
The purpose of this project is to make it easier for people that attend social
gatherings or events to select a songs and form their own playlist according to
the mood or perference of the attendees. The current implementation (PlayMyWay)
has an unflattering and difficult to use UI, as well as no easy way for the
average person to integrate the software into their party. We plan on making a
revised version that has an elengant web app interface, and an easy to install
server.

Social gatherings are much more enjoyable when most of the attendee's enjoy
the music that is being played. This project was inspired by \\

Social gatherings are much more enjoyable when most of the attendee's enjoy
the music that is being played. This project was inspired by

\subsection{The Stakeholders}

\subsubsection{The Client}

\textbf{Event Organizer}\\
The client is the host of the social event or gathering, who is trying to save
money by not hiring a DJ and simply relying on this system, which allows the
attendees to chose what songs they want to listen to. Having users select music
will put less stress on the event organizers, and allow them to focus on other
aspects of the event, or let them enjoy themselves. \\ \textbf{DJ} \\ The client can also be a DJ, hired by a party planner, who is not willing to put up with people fighting over what song to play next.

\subsubsection{The Customers}

\textbf{Event Attendee's}\\
If the system is working properly and attendee's are voting for songs, then
the event will be more enjoyable for them.
\subsubsection{Other Stakeholders}


No other stakeholders have been discovered.

\subsection{Mandated Constraints}

\begin{itemize}
\item The front-end of the product will take the form of a web-app that can
run on any javascript-enabled browser. This means the app should be able to
accommodate all common Operating Systems (phone and desktop) and all
common Javascript-enabled browsers. Internet Explorer may be exempt
\item The front-end webapp and server need to both be connected to the same
WiFi network.
\item The server needs to be stable with very low downtime, in order to prevent
scenarios where the music stops playing accidentally.
\end{itemize}

\subsection{Naming Conventions and Terminology}

\begin{table}[!h]
\centering
\begin{tabular}{ c | c }
\textbf{Shorthand} & \textbf{Explanation}\\ \hline
JS & Javascript\\
HTML & Hypertext Markup Language\\
CSS & Cascading Style Sheets\\
Event & Any event that includes shared music listening. Party, Wedding, etc. \\
UI & User Interface\\
\end{tabular}
\end{table}
\subsection{Relevant Facts and Assumptions}

\begin{itemize}
\item We assume that users of the app are impatient (they don't want to spend a
long time learning the software)
\end{itemize}

\section{Functional Requirements}
 \begin{itemize}
 \item The web page created in HTML and Javascript will display graphics and other information to user like title.
 \item The Javascript will fetch the five options for music playback and will display them on the user's browser.
 \item The web page should also keep track of the number of votes for any song.
 \item The html page should also include a graphic where the user can place their vote.
 \item The html page should also only list valid options for music (should only list songs stored on the server).
 \item Cookies will also be created to only allow one vote for the user.
 \item After the song has been selected to play then the html page should update to only show songs that haven't been played and should reset votes.
 \item The server should pick 5 random unique songs until no unique songs are left and repeat process.
 \item The server should total votes and share the information with the user through the html page.
 \item The server should sort the songs after voting and play the most voted song.
 \item The cookie should also remember what song the user picked and then reset after voting has reset.
 \end{itemize}

\subsection{The Scope of the Work and the Product}

\subsubsection{The Context of the Work}
\begin{figure}[!ht]
  \caption{A diagram of the context of work}
  \centering
    \includegraphics[width=0.5\textwidth]{contextWork.png}
\end{figure}
\subsubsection{Work Partitioning}

\subsubsection{Individual Product Use Cases}

\subsection{Functional Requirements}
\begin{itemize}
\item The web page created in HTML and Javascript will display graphics and other information to user like title.
\item The Javascript will fetch the five options for music playback and will display them on the user's browser.
\item The web page should also keep track of the number of votes for any song.
\item The html page should also include a graphic where the user can place their vote.
\item The html page should also only list valid options for music (should only list songs stored on the server).
\item Cookies will also be created to only allow one vote for the user.
\item After the song has been selected to play then the html page should update to only show songs that haven't been played and should reset votes.
\item The server should pick 5 random unique songs until no unique songs are left and repeat process.
\item The server should total votes and share the information with the user through the html page.
\item The server should sort the songs after voting and play the most voted song.
\item The cookie should also remember what song the user picked and then reset after voting has reset.
\end{itemize}

\section{Non-functional Requirements}

\subsection{Look and Feel Requirements}
\subsubsection{Appearance Requirements}
\subsubsection{Style Requirements}
\begin{itemize}
\item Visually appealing
\item Webapp design conforms to material design standards
\item No noticeable UI 'lag'
\end{itemize}

\subsection{Usability and Humanity Requirements}
\subsubsection{Ease of Use Requirements}
\subsubsection{Personalization Requirements}
\subsubsection{Learning Requirements}
\begin{itemize}
\item Straight-forward web app
\item New users should be able to learn and adopt it easily
\item No sign up required
\item Automatically connects to server with little to no input from the user
\end{itemize}

\subsection{Performance Requirements}
\subsubsection{Speed and Latency Requirements}
\subsubsection{Precision and Reliability Requirements}
\subsubsection{Longevity Requirements}

\begin{itemize}
\item Songs should play one after another with little to no delay in between
\item Server should have low downtime
\item Robust error handling
\end{itemize}

\subsection{Operational and Environmental Requirements}

\begin{itemize}
\item Server runs on a Raspberry PI device
\item Webapp runs on all javascript-enabled web browsers
\item Emphasis placed on phone browsers
\item System must be able to handle
\end{itemize}

\subsection{Maintainability and Support Requirements}

\begin{itemize}
\item Minimal required maintenance
\item Automatically pushed security updates
\end{itemize}

\subsection{Security Requirements}

\begin{itemize}
\item A user should only be allowed to vote once per \textit{'round'} of votes
\item Restarting the app/WiFi/phone should not change the fact that a user can
only vote once.
\item Only users attending the event should be allowed to vote
\item No private server data should be visible to anyone but the administrator
\item No private data about other users should be visible to anyone but the
administrator
\end{itemize}

\subsection{Cultural Requirements}
There are no cultural requirements for this project.
\subsection{Legal Requirements}
\subsubsection{Compliance Requirements}
- Music that is played should have legal rights to be played publically \\
- The open project we are recreating has an open MIT license \href{LICENSE.txt} {that can be found here}\\
\subsubsection{Standards Requirements}
There are none....

\subsection{Health and Safety Requirements}


This section is not in the original Volere template, but health and safety are
issues that should be considered for every engineering project.

\section{Project Issues}

\subsection{Open Issues}

The most important issue right now is how exactly do we make sure that a user can only vote once per song, without requiring people to sign up for an account. We also want the user to be able to change their vote multiple times when selecting the next song and it is in queue.

\subsection{Off-the-Shelf Solutions}

The project that we are modeling (PlayMyWay) already does most of what our project will do.

As for other solutions, there are various libraries that we will use in our
development. These libraries will help with various functionality, and include
(but are not limited to):
\begin{itemize}
\item Express.JS (Server framework for Node.JS)
\item Angular.JS (Front-end framework for the webapp)
\item nodeunit (Unit testing package)
\item Mocha (general testing package)
\end{itemize}
\subsection{New Problems}

DJ.Js is based off the open source project called \href{https://github.com/malithsen/playmyway}{PlayMyWay that can be found here.} We are going to recreate the project, in javascript. The original project was written in Jade, which is a Object Oriented programming language based on Java. Jade is kind of outdated and not as universal as javascript, which is the golden standard for web page applications. We don't need any new installations just a device that can access the internet and has a internet browser that runs javascript.

\subsection{Tasks}

We are given an outline of the things we need for this project which includes a proof of concept, testing plan, a design, and a final presentation. The proof of concept will be early in the project but will allow us to make a very basic outline of the project like be able to get requests from a device to a server vice versa. The test plan will tell us how we are going to test our product so we know whether or not the project has successfully fulfilled our requirements The design of the project will be more specific to coding and will come later but can be broken down in these simple steps:
\begin{enumerate}
\item Build a node.js sever using Amandeeps Raspberry Pi for the
\item Build the UI using javascript
\item Testing (unit,general testing)
\end{enumerate}
The Final Presentation will be the last step in the project when it is complete and fully functional.
\subsection{Migration to the New Product}

There is no transition needed because we are recreating a web app, that already exists.

\subsection{Risks}

Many risks can occur while trying to implement our product which include:
\begin{itemize}
\item The product is a webapp and therefore relies on an internet connection
\item Bugs or catastrophic errors in the server could cause the music to start playing sporadically
\item Only people with a device that can connect to the internet can use the web app
\end{itemize}

\subsection{Costs}

There will be no monetary costs, the music we are using to test our webapp will be unlicensed and free to play, and because the project we are recreating is an open source project which we are free to use in any way. The only other cost is the amount of effort and time we put in the project which should not be more than 100 hours, but could take longer.

\subsection{User Documentation and Training}

We will integrate a small help button at the end of the webpage that will describe how the webpage works. After this short tutorial the user should know how the website works.

\subsection{Waiting Room}

Something that is part of our vision is having all genres of music, having a diverse list of music so users can select songs that cater to their tastes. We also want to add "moods" that correspond to certain event you are in, so a wedding would be more of a happy mood, where as a party will have a Festival will have a celebratory mood.

\subsection{Ideas for Solutions}

\newpage

\bibliographystyle{plainnat}

\bibliography{SRS}

\newpage

\section{Appendix}

This section has been added to the Volere template.  This is where you can place
additional information.

\subsection{Symbolic Parameters}

The definition of the requirements will likely call for SYMBOLIC\_CONSTANTS.
Their values are defined in this section for easy maintenance.


\end{document}
